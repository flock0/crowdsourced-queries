\documentclass{article}

\usepackage[utf8]{inputenc}
\usepackage{setspace}


\begin{document}

\begin{titlepage}
 \vspace*{\fill}
\begin{center}
\begin{spacing}{2.0}

%Title
{\huge Crowdsourced Query Understanding and Optimization}
\\[1.5cm]

%Course - Year
\textsc{ \Large Big Data Course, EPFL \\ 2015}
\\[1.5cm]

%Authors
{\Large Florian Chlan, François Farquet, Joachim Hugonot, Simon Rodriguez, Kristof Szabo, Florian Vessaz, Guo Xinyi, Vincent Zellweger}
\\[0.5cm]

%Supervisor
{\large T.A. : Immanuel Trummer}

\end{spacing}
\end{center}
\vspace*{\fill}
\end{titlepage}

\tableofcontents
\newpage

\section{Abstract}

\section{Crowdsourcing generalities}



\section{Our work} % < temporary title
\subsection{Global structure}
\subsection{Query language and parsing}
\subsection{AMT communication}
\subsection{Implementation choices}
\begin{itemize}
\item For a given query, we decompose it into sub-tasks which are generally really close to the semantic, meaning if we have a "WHERE" in the query we will have a dedicated task to handle this "WHERE".
\item We \textbf{always} start with the "FROM" clause, where we ask \textbf{one} worker to give either a website either or a list of primary keys. After that we can continue the execution of this query by using the results given in this first step.
\item For each task we try to use simple and understandable formulation, as well as proper description and keywords for a better visibility on the mturk platform.

\end{itemize}

\subsection{User interface}
\subsection{Performance Analysis}

\section{Future improvements} % < maybe move it into Results ?

\section{Feedback/Work repartition} % < personal feedback
\begin{itemize}
\item Florian Chlan
\item François Farquet
\item Xinyi Guo
\item Joachim Hugonot
\item Simon Rodriguez
\item Kristof Szabo
\item Florian Vessaz
\item Vincent Zellweger
\end{itemize}
\section{Conclusion}

\end{document}
