\documentclass{article}

\usepackage[utf8]{inputenc}
\usepackage{setspace}


\begin{document}

\begin{titlepage}
 \vspace*{\fill}
\begin{center}
\begin{spacing}{2.0}

%Title
{\huge Crowdsourced Query Understanding and Optimization}
\\[1.5cm]

%Course - Year
\textsc{ \Large Big Data Course, EPFL \\ 2015}
\\[1.5cm]

%Authors
{\Large Florian Chlan, François Farquet, Joachim Hugonot, Simon Rodriguez, Kristof Szabo, Florian Vessaz, Guo Xinyi, Vincent Zellweger}
\\[0.5cm]

%Supervisor
{\large T.A. : Immanuel Trummer}

\end{spacing}
\end{center}
\vspace*{\fill}
\end{titlepage}

\tableofcontents
\newpage


\section{Introduction}



\section{Our work} % < temporary title
\subsection{Global structure}
We have 3 parts in the project, the part were we parse the query and create the associated tree, the part were we use this tree to create HIT and finally the part which communicate with the Amazon Mechanical Turk platform (mturk) and send those HIT and retrieve answers.\\
To visualise the status of our queries, we have a user-friendly interface on which we can find relevant information for each query.\\ The majority of the code is written in Java and Scala but we also have HTML and CSS for the interface
\subsection{Query language and parsing}
\subsection{Usage of Amazon Mechanical Turk}
We use mturks huge, scalable workforce to help with retrieving data for and processing the query. In our implementation, we currently don't use qualifications. This means that the answers from each and everyone of the workers is treated equally.
\subsection{Implementation choices}
\begin{itemize}
\item For a given query, we decompose it into sub-tasks which are generally really close to the semantic, meaning if we have a "WHERE" in the query we will have a dedicated task to handle this "WHERE".
\item We \textbf{always} start with the "FROM" clause, where we ask \textbf{one} worker to give either a website or a list of primary keys. After that we can continue the execution of this query by using the results given in this first step.
\item For each task we try to use simple and understandable formulation, as well as proper description and keywords for a better visibility on the mturk platform. 

\end{itemize}

\subsection{User interface}
\subsection{Performance Analysis}

\section{Future improvements} % < maybe move it into Results ?
Possibilities of improvement for this project are countless but the following point are the most relevant for the moment by order of importance:
\begin{itemize}
\item For the very first task, the "FROM" task, add a verification phase afterwards because this step is critical, if the website returned is not good, we won't be able to have any satisfying results.
\item Implement a majority vote for some of the questions. The results will be more precise but also more expensive, as more workers need to be asked the same questions. Using a majority vote for multiple choice or true/false questions is trivial. For cases where free text answers or numeric answers are expected, implementing a majority vote will be more sophisticated, as the domain of possible answers is sheer endless.
\item Implement more operators.
\item Use qualifications to get more reliable results. This would mean that a set of sample questions has to be created, for which the answers are already known. Workers are then at first asked to answer the sample questions to proof their seriousness and then the real questions afterwards. It is then possibly to rank, weight or filter workers based on their performance and this would ultimately lead to better and more precise results.


\end{itemize}
Another great feature to add, but not so simple to implement, is the caching of previous results, indeed if you retrieve results from a cache, rather than asking again on mturk, the result will be instantaneous, free and accurate.
\section{Feedback/Work repartition} % < personal feedback
\begin{itemize}
\item Florian Chlan
\item François Farquet
\item Xinyi Guo
\item Joachim Hugonot
\item Simon Rodriguez
\item Kristof Szabo
\item Florian Vessaz
\item Vincent Zellweger
\end{itemize}
\section{Conclusion}

\end{document}
