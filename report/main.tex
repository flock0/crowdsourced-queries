\documentclass{article}

\usepackage[utf8]{inputenc}
\usepackage{setspace}


\begin{document}

\begin{titlepage}
 \vspace*{\fill}
\begin{center}
\begin{spacing}{2.0}

%Title
{\huge Crowdsourced Query Understanding and Optimization}
\\[1.5cm]

%Course - Year
\textsc{ \Large Big Data Course, EPFL \\ 2015}
\\[1.5cm]

%Authors
{\Large Florian Chlan, François Farquet, Joachim Hugonot, Simon Rodriguez, Kristof Szabo, Florian Vessaz, Guo Xinyi, Vincent Zellweger}
\\[0.5cm]

%Supervisor
{\large T.A. : Immanuel Trummer}

\end{spacing}
\end{center}
\vspace*{\fill}
\end{titlepage}

\tableofcontents
\newpage


\section{Introduction}



\section{Our work} % < temporary title
\subsection{Global structure}
We have 3 parts in the project, the part were we parse the query and create the associated tree, the part were we use this tree to create HIT and finally the part which communicate with the Amazon Mechanical Turk platform (mturk) and send those HIT and retrieve answers.\\
To visualise the status of our queries, we have a user-friendly interface on which we can find relevant information for each query.\\ The majority of the code is written in Java and Scala but we also have HTML and CSS for the interface
\subsection{Query language and parsing}
The parsing is inspired by the SQL query. We tried to include a lot a feature given by SQL into out language, such as the ORDER BY, the WHERE and the JOIN. To be more flexible we have introduced two main thing. First a hierarchy on the different command and the second ont is that the query is recursively caling. It mean that the query is parsed with query and sub query.
For most of the different step of the hierarchy, that are optional element and we could avoid them.
For the explanation of the query, we will use the following definition
\begin{itemize}
\item Q1, Q2, ... are different query
\item E is an element
\item Int is a number
\item Str is a string. The string has to be placed between " and is allowed to contain letter (capital or normal), integer and \_
\item NL is some natural language. The natural language has to be inserter into braket [ ] and is allowed to contain letter (capital or normal), integer and space.
\item Bool is the boolean type. It could be either True od False
\end{itemize}
\subsubsection{Root Node}
The first hierarchy level is all the different link from a query with another query
\begin{itemize}
\item JOIN ON: Is a junction between two query that join the information of the first and the second information given a specific attribute.
\begin{verbatim}
(Q1) JOIN (Q2) ON E
\end{verbatim}
\item IN: Return all the different row of Q1 that are in Q2 too
\begin{verbatim}
(Q1) IN (Q2)
\end{verbatim}
\item NOT IN: Return all the different row of Q1 that are not in Q2
\begin{verbatim}
(Q1) NOT IN (Q2)
\end{verbatim}
\item INTERSECT: Return all the different row that are both in Q1 and in Q2 (give the same result as the IN 
\begin{verbatim}
(Q1) INTERSECT (Q2)
\end{verbatim}
\item UNION: Return all the different row of Q1 plus the different row of Q2
\begin{verbatim}
(Q1) UNION (Q2)
\end{verbatim}
\item Nothing: If we don't need to use a link with another query
\begin{verbatim}
Q1
\end{verbatim}
\end{itemize}
\subsubsection{Limit Node}
We could or not ask for a limit on the number of item that we want to return.
\begin{itemize}
\item LIMIT: return a specific number of element
\begin{verbatim}
(Q1) LIMIT Int
\end{verbatim}
\item Nothing:  If we don't need to limit the number of rows
\begin{verbatim}
Q1
\end{verbatim}
\end{itemize}
\subsubsection{Order Node}
This node is used to order the result on a specific order. The hierarchy is created such that to be able to limit the X first row on a specific order. It won't be logic to have an order on an aleatory order
\begin{itemize}
\item ORDER BY: Order the row on one or more element on a specific order
\begin{verbatim}
(Q1) ORDER BY E [ASC|DESC] [, E (ASC|DESC)]*
\end{verbatim}
The order could be on ascending (ASC) or descending (DESC) order, and it is possible to order on more than one element
\item Nothing:  If we don't need to order the rows
\begin{verbatim}
Q1
\end{verbatim}
\end{itemize}
\subsubsection{Group Node}
Thie node is used to group the different result row given a specific element. It could be usefull for SUM or other element like this one. The hierarchy is ceated such that because the LIMIT and the ORDER has no impact on the GROUP BY but the GROUP BY give another way to order
\begin{itemize}
\item OGROUP BY: Ask to group the different row on a specific order
\begin{verbatim}
(Q1) GROUP BY E
\end{verbatim}
\item Nothing:  If we don't need to group the rows
\begin{verbatim}
Q1
\end{verbatim}
\end{itemize}
\subsubsection{Where Node}
This node is used to remove some row that don't match some specific condition. The condition is also parsed by the system and is explain in a subsection. The hierarchy is here a little bit tricky and don't give us as possibility as SQL, since we couldn't put condition on the sum of element that came after the group by. We choose to implement the WHERE before the GROUP BY because in most of the case, we will group element that match a condition, and not match a condition on grouped element.
It is one of the place in the parsing part where optimisation could ocure.
\begin{itemize}
\item WHERE: The bunch of condition that has to occure
\begin{verbatim}
(Q1) WHERE Condition
\end{verbatim}
\item Nothing:  If we don't need to ask a specific condition
\begin{verbatim}
Q1
\end{verbatim}
\end{itemize}
\paragraph{Condition}
The bunch of condition could accept different kind of stuff. This different condition could be alone, with another, recursive or with natural language.
\begin{itemize}
\item < : Ask an element to be smaller than a condition. The condition has to be a number or a natural language
\begin{verbatim}
E < [Int|NL]
\end{verbatim}
\item > : Ask an element to be bigger than a condition. The condition has to be a number or a natural language
\begin{verbatim}
E < [Int|NL]
\end{verbatim}
\item <= : Ask an element to be smaller or equals than a condition. The condition has to be a number or a natural language
\begin{verbatim}
E < [Int|NL]
\end{verbatim}
\item >= : Ask an element to be bigger or equals than a condition. The condition has to be a number or a natural language
\begin{verbatim}
E < [Int|NL]
\end{verbatim}
\item = : Ask an element to be equals than a condition. The condition could be a lot of ellement. It could be a number, a string, some natural language or a new sub queries. Attention, the natural language and the query has to return only one element (one item and one rows) to be consistant. It is not possible to match the equality of an element with a bunch of data.
\begin{verbatim}
E = [Int|NL|Bool|Q1]
\end{verbatim}
\item IN : Ask an element to be inside a list of ellement. It is quite the same as = since the data could be in a set of element. Attention, some restriction may still apply. Since the query or the natural language are allowed to give a set of data, it has to give a set of single data, not couple or something else. For example, it is possible to check a name in a set of name, not in a set of couple (name, first name).
\begin{verbatim}
E IN [([Int|NL|Bool|Q1] [, [Int|NL|Bool|Q2]]*|NL|Q3]
\end{verbatim}
The list of element could be
\begin{itemize}
\item an explicit set
\begin{verbatim}
 (cond1, cond2, cond3, ...)
\end{verbatim}
\item some natural language
\item another query
\end{itemize}
\item NOT IN : Exactly the same functionality as IN but ask the element to be not in the set.
\begin{verbatim}
E NOT IN [([Int|NL|Bool|Q1] [, [Int|NL|Bool|Q2]]*|NL|Q3]
\end{verbatim}
\item AND : The parser will parse this part recursively. It ask to have the condition C1 and the condition C2 to be satisfied to accept the condition.
\begin{verbatim}
(C1) AND (C2)
\end{verbatim}
\item OR : The parser will parse this part recursively. It ask to have the condition C1 or the condition C2 to be satisfied to accept the condition.
\begin{verbatim}
(C1) OR (C2)
\end{verbatim}
\end{itemize}
\subsubsection{Select Node}
This part is the final part of the node, where we will select the different element that we want. It could be either only natural language, a selection of field in a natural language question or a selection of field on a subqueries.
Since the element could be quite complexe, we will resume it has Elements in the definition and explain them more specificly on another section.
\begin{itemize}
\item SELECT FROM NL: Has to extract the specified field from the question asked in natural language.
\begin{verbatim}
SELECT Elements FROM NL
\end{verbatim}
\item SELECT FROM sub queries: Has to extract the specified field from a subqueries that will be computed from the root node.
\begin{verbatim}
SELECT Elements FROM (Q1)
\end{verbatim}
\item NL: Ask something only in natural language
\begin{verbatim}
NL
\end{verbatim}
\end{itemize}
\subsubsection{Elements Node}
First of all, the differents elements in the node element has to be a list of ellement. This list is presented like that
\begin{verbatim}
(E1, E2, E2, ....)
\end{verbatim}
We could ask different informations about this ellement. It means:
\begin{itemize}
\item The sum of the different ellements (given a group by)
\begin{verbatim}
SUM(E)
\end{verbatim}
\item Have only distinct row, to avoid having ten time the same one (for example after an union or a bad worker)
\begin{verbatim}
DISTINCT(E)
\end{verbatim}
\end{itemize}
And finaly the tupple could be of two different kind
\begin{itemize}
\item numeric tupple
\begin{verbatim}
NUMERIC E
\end{verbatim}
\item normal tupple
\begin{verbatim}
E
\end{verbatim}
\end{itemize}
\subsubsection{Remarque}
During our test, we found some probleme with this parsing. First of all the using of the parenthesis before and after the different ellement is disgrassius. The main point of it is that the recursive parsing create some infinite loop (was looking for the fist element on the root node, that was recursive for JOIN and look for the first element of root note, ...). So we had to add the parenthesis that remove the first element when askin gthe recursive function.
Another problem that we tried to solve is that if we ask a query without the parenthesis, it seems to parse it corectly but forget the end of the query. For example
\begin{verbatim}
SELECT (name) FROM (king of France) ORDER BY age_of_death
\end{verbatim}
parsed well the SELECT and the FROM but seems to ignore the ORDER BY
Another problem is that when we try to parse too long query, the parser seems to stop working correctly. For example, a query using GROUP BY, ORDER and LIMIT seems to work well, but if we insert it in a JOIN, it crash because it need the WHERE, which is not needed.
\subsection{Usage of Amazon Mechanical Turk}
We use mturks huge, scalable workforce to help with retrieving data for and processing the query. In our implementation, we currently don't use qualifications. This means that the answers from each and everyone of the workers is treated equally.\\
To interact with Amazon Mechanical Turk service, we use the available GET interface. We developed our own AMT communicator, in charge of interpreting our internal abstract representation of a task (a HIT) to generate the corresponding URL request. We provide a Question abstract class and a number of subclasses to represent a wide range of question types : text answer, URL answer, boolean answer,...
\subsection{Implementation choices}
\begin{itemize}
\item For a given query, we decompose it into sub-tasks which are generally really close to the semantic, meaning if we have a "WHERE" in the query we will have a dedicated task to handle this "WHERE".
\item We \textbf{always} start with the "FROM" clause, where we ask \textbf{one} worker to give either a website or a list of primary keys. After that we can continue the execution of this query by using the results given in this first step.
\item For each task we try to use simple and understandable formulation, as well as proper description and keywords for a better visibility on the mturk platform. 
\item The parallelization has been implemented with Scala Futures. The result of each HIT is contained in a Future. As soon as the result arrives, the onSuccess method of the Future is called sending a new HIT if possible. For example, if we send a query containing SELECT and WHERE tasks, some WHERE tasks can be submited even if not all the select tasks are finished.
\item Parallelization of JOIN tasks work a bit differently. The two branches of the JOIN are executed in parallel but the JOIN task is waiting for the results of both parts before beginning. Changing this is a possible improvement for the future.
\end{itemize}

\subsection{User interface}
To visualize the status of the queries, we design this user  interface. The interface always transmit queries to the server then get JSON object of the query results from the server, and show the results rationally to the user. The implementation is in Javascript and HTML. We use the Bootstrap v3.3.4 
 \emph{(http://getbootstrap.com)} as reference to make the interface well-formed.\\
 Users could enter their query by enter the textbox but should be in agreement with the query grammatical rules, otherwise, after submitting the query, you get a parsing failed remind. We also have an easy version of generating queries for non-specialists. Clicking on the Assisted query creation button, a form with basic operation like SELECT, JOIN, WHERE popped out, the user just needs to fill in the blanks in the form.\\
 After you successfully generating a query, you can see the Querying ID,  Querying ID,  status, Number of results, Start time, Number of results, Start time, Finish time. If you generate many queries, there is a list of queries show in the screen. You can refresh the list by clicking on the button. The list result also automatically refreshed every 15000 sec. \\
By clicking on the area of the query, the corresponding query detail results will popped out. In the detail window, you could see the query string, the final query result. You can also see the subtasks' information, like subtask Task ID, Status, Operator, Generated Hits, Completed hits and every subtask results. If you are not interested in the query any more, you can easily abort the query by clicking the Abort button at the end of the window.
\subsection{Performance Analysis}
%Increasing price is good, but we still have huge delay. Some implementation choices might be expanded in the future (WHERE applied after the SELECT when there is a LIMIT leads to a lower number of results, for instance).
%Phrasing of questions can alter the results from the workers.
%Accuracy : depends on the first steps (FROM fi)
\section{Future improvements} % < maybe move it into Results ?
Possibilities of improvement for this project are countless but the following point are the most relevant for the moment by order of importance:
\begin{itemize}
\item For the very first task, the "FROM" task, add a verification phase afterwards because this step is critical, if the website returned is not good, we won't be able to have any satisfying results.
\item Implement a majority vote for some of the questions. The results will be more precise but also more expensive, as more workers need to be asked the same questions. Using a majority vote for multiple choice or true/false questions is trivial. For cases where free text answers or numeric answers are expected, implementing a majority vote will be more sophisticated, as the domain of possible answers is sheer endless.
\item Implement more operators.
\item Use qualifications to get more reliable results. This would mean that a set of sample questions has to be created, for which the answers are already known. Workers are then at first asked to answer the sample questions to proof their seriousness and then the real questions afterwards. It is then possibly to rank, weight or filter workers based on their performance and this would ultimately lead to better and more precise results.


\end{itemize}
Another great feature to add, but not so simple to implement, is the caching of previous results, indeed if you retrieve results from a cache, rather than asking again on mturk, the result will be instantaneous, free and accurate.
\section{Feedback/Work repartition} % < personal feedback
\begin{itemize}
\item Florian Chlan
\item François Farquet
\item Xinyi Guo
\item Joachim Hugonot
\item Simon Rodriguez
\item Kristof Szabo
\item Florian Vessaz
\item Vincent Zellweger
\end{itemize}
\section{Conclusion}

\end{document}
